\documentclass[conference,compsoc]{IEEEtran}
\usepackage{datetime}
\usepackage{caption}
\usepackage{listings}
\usepackage{cite} 
\usepackage{hyperref}
\usepackage{xcolor}

\begin{document}

% paper title
\title{GreenIT Article Review Report on\\
"How realistic are claims about the benefits of using digital technologies for GHG emissions mitigation?"
\\

%current time
{\small \today~-~\currenttime}}
 
% student name
\author{\IEEEauthorblockN{TITCHEU YAMDJEU Pierre Wilfried}
\IEEEauthorblockA{University of Luxembourg\\
Email: pierre.titcheu.001@student.uni.lu}
\\
($\pm$ 5 pages)\\}

% make the title area
\maketitle

% abstract
\begin{abstract}

This report provides a critical review of the paper ”How realistic are claims about the benefits of using digital technologies for GHG emissions mitigation?” by Aina Rasoldier et al. The paper evaluates the claims about digital solutions, particularly carpooling platforms, in reducing greenhouse gas (GHG) emissions, proposing guidelines for more realistic assessments.

\end{abstract}

% Context
\section{Introduction ($\pm 0,5p$)}

The reviewed paper by Aina Rasoldier et al. explores
the potential of digital technologies, particularly carpooling
platforms, in mitigating greenhouse gas (GHG) emissions.
Set against the backdrop of increasing digitalization and its
environmental implications, the paper critically examines the
prevailing assumptions about the efficacy of digital solutions
in GHG reduction. The importance of this research lies in
its relevance to GreenIT, a domain that emphasizes the role
of technology in addressing environmental challenges. The
paper aligns with the GreenIT perspective by scrutinizing the
real-world impacts of digital technologies on the environment,
underscoring the need for comprehensive evaluations beyond
mere technological advancements.

% Research Problem 
\section{Research problem addressed ($\pm 0,5p$)}

The paper addresses the research problem of evaluating
the actual impact of digital technologies on GHG emissions.
It questions the prevailing optimistic claims regarding the
efficacy of digital solutions, specifically carpooling platforms,
in environmental conservation. The research objectives include
establishing a framework for a more realistic and comprehen-
sive assessment of digital solutions in GHG mitigation. The
paper hypothesizes that current evaluations of digital tech-
nologies for GHG reduction are often overestimated, lacking
in critical analysis of life-cycle impacts, structural impacts,
and rebound effects. It aims to substantiate this hypothesis
by applying its proposed evaluation framework to the case of
carpooling platforms.

% Paper references review 

\section{Extended Literature Review (±1p)}

The paper "How realistic are claims about the benefits of using digital technologies for GHG emissions mitigation?" by Aina Rasoldier et al. presents an in-depth analysis of the intersection between digital technologies and their role in mitigating greenhouse gas (GHG) emissions. This literature review delves deeper into the various sources and arguments presented in the paper, reflecting the complexity and multifaceted nature of the topic.

At the core of the paper's argument is the relationship between digital technologies and the environment, specifically focusing on GHG emissions. The review begins by contextualizing this relationship within historical environmental movements and policies. It references the United Nations Stockholm Conference on the Human Environment (1972) and the establishment of the UNEP, highlighting how global environmental consciousness has evolved over decades.

The Brundtland Report of 1987 is another critical piece of literature mentioned, which introduced the concept of sustainable development. This report's definition of sustainable development as meeting the needs of the present without compromising future generations is particularly relevant to the paper's focus on digital technologies. It underscores the importance of developing technologies that are environmentally sustainable.

The Sustainable Development Goals (SDGs) of the United Nations, especially SDG 13 - Climate Action, form a backdrop against which the paper evaluates digital solutions. The SDGs provide a globally recognized framework for assessing the effectiveness and alignment of digital innovations with environmental targets.

In its exploration of environmental assessments, the paper draws upon methodologies like Life Cycle Assessment (LCA), as defined by ISO 14040 and ISO 14044. LCA offers a comprehensive approach for evaluating the environmental impact of technologies, encompassing their entire life cycle. This methodology is crucial for understanding the full environmental footprint of digital solutions, from production to disposal.

The European Green Deal is another pivotal point of reference, reflecting the policy landscape shaping the development and deployment of digital technologies. The Green Deal's emphasis on achieving climate neutrality by 2050 and integrating digital strategies into environmental policies aligns with the paper's exploration of digital solutions for GHG mitigation.

A significant aspect of the literature review is the discussion around Green IT and IT for Green. This distinction is crucial for understanding the dual role of digital technologies. Green IT focuses on the sustainability of IT systems themselves, while IT for Green emphasizes the use of IT in promoting broader environmental sustainability.

Furthermore, the paper touches upon the challenges and considerations involved in assessing the benefits of digital solutions for GHG mitigation. It critically examines the scenarios and methodologies used in existing literature, highlighting the need for detailed and context-specific evaluations. This includes an analysis of the potential overestimation of digital solutions' benefits and the importance of integrating them into realistic and comprehensive strategies for GHG reduction.

In summary, the extended literature review in Rasoldier et al.'s paper presents a thorough exploration of the multifaceted relationship between digital technologies and environmental sustainability. By integrating historical perspectives, methodological frameworks, policy contexts, and critical analyses of existing literature, the review provides a comprehensive foundation for understanding and evaluating the claims about the benefits of digital technologies for GHG emissions mitigation.

% Contributions 
\section{Scientific contributions w.r.t. GreenIT and IT for Green ($\pm 2p$)}

\subsection{Introduction to the Interplay between GreenIT and IT for Green}

The paper "How realistic are claims about the benefits of using digital technologies for GHG emissions mitigation?" by Aina Rasoldier et al. provides a comprehensive analysis of the role of digital technologies in environmental sustainability. This analysis bridges the concepts of GreenIT, which focuses on the sustainability of information technology itself, and IT for Green, which emphasizes the use of IT to support environmental sustainability across various sectors.

\subsection{Detailed Examination of GreenIT}

\subsubsection{Life Cycle Assessment and Sustainability of ICT}

The paper delves into the Life Cycle Assessment (LCA) of digital technologies, echoing the ISO standards 14040 and 14044 discussed in the lecture. This assessment extends from the manufacturing phase to the disposal of IT systems, addressing the entire spectrum of environmental impacts associated with ICT. The paper’s focus on the life-cycle impacts highlights the importance of considering the direct effects of providing ICT services, including the production, use, transport, and end-of-life stages. This comprehensive approach aligns with the  emphasis on multi-criteria assessment methods that evaluate multiple environmental impact indicators, including GHG emissions, resource consumption, and waste generation.

\subsubsection{Green Standards and Metrics in ICT}

Further contributing to the understanding of GreenIT, the paper discusses the need for robust methodologies to measure the environmental impacts of ICT. This includes a detailed analysis of GHG emissions and energy use, which are pivotal metrics in assessing the sustainability of IT systems. The paper's approach to evaluating these metrics is consistent with the  discussion on the main metrics for GreenIT, such as primary energy consumption and greenhouse gas emissions. By scrutinizing these metrics, the paper enhances our understanding of the environmental footprint of digital solutions, an aspect crucial to the development of sustainable IT systems.

\subsection{Contributions to IT for Green}

\subsubsection{Digital Solutions in Mitigating GHG Emissions}

In the realm of IT for Green, the paper explores how digital technologies can aid in achieving environmental sustainability goals, particularly in the context of SDG 13 - Climate Action. It evaluates the potential of digital solutions in reducing GHG emissions across various domains, including transportation, energy, and agriculture. This evaluation is in line with the  focus on leveraging IT to support specific Sustainable Development Goals (SDGs).

\subsubsection{Balancing GreenIT and IT for Green}

The paper acknowledges the potential trade-offs between the environmental impact of developing and using digital technologies (GreenIT) and the benefits these technologies can offer in terms of environmental sustainability (IT for Green). This nuanced perspective recognizes the complexities in aligning IT with environmental goals and reflects the concepts covered in the lecture on the interplay between GreenIT and IT for Green.

\subsection{Evaluation of Digital Technologies' Impact on GHG Mitigation}

\subsubsection{Critical Analysis of Claimed Benefits}

The paper critically examines the often-stated benefits of digital technologies in mitigating greenhouse gas (GHG) emissions. It probes the validity of these claims by exploring various scenarios and hypotheses underlying these assessments. This aligns with the  emphasis on the importance of a rigorous and transparent evaluation process in GreenIT, where claimed benefits must be substantiated with credible and verifiable data.

\subsubsection{Methodological Rigor in Assessments}

In line with the  focus on methodological rigor, the paper proposes guidelines for studies evaluating the impact of digital solutions on GHG emissions. These guidelines stress the importance of life-cycle analysis, consideration of structural impacts, and the connection of proposed solutions with global sustainability goals. This approach is crucial for ensuring that the evaluations of digital technologies' contributions to GHG reduction are comprehensive and grounded in reality.

\subsection{Implications for Policy and Strategy}

\subsubsection{Influence on Regulations and Policies}

The paper's findings have implications for the regulation of the ICT sector and the development of digital strategies aimed at environmental sustainability. By providing a clear assessment of the environmental impacts of digital technologies, the paper supports the  discussion on the need for informed policy-making in GreenIT and IT for Green. This is particularly relevant in the context of the European Union's Green Deal and the UN's Sustainable Development Goals, which prioritize sustainability in technology development and use.

\subsubsection{Identifying Research Gaps and Future Directions}

The paper also identifies overlooked research directions in the field of GreenIT and IT for Green. This is in harmony with the  call for continuous exploration and innovation in the field. By pointing out the gaps in current research, the paper sets the stage for future studies that can further enhance our understanding of the interplay between ICT and environmental sustainability.

\subsection{Real-world Applications and Case Studies}

\subsubsection{Practical Insights into GreenIT Strategies}

The paper provides practical insights into how digital technologies can be implemented in real-world scenarios to achieve environmental benefits. This practical approach resonates with the  focus on applying GreenIT strategies in diverse contexts, such as energy-efficient computing and sustainable data centers.

\subsubsection{Case Study Analysis}

By examining specific case studies, such as the role of digital platforms in carpooling and its impact on GHG emissions, the paper offers concrete examples of how IT can contribute to environmental sustainability. These case studies illustrate the  emphasis on the practical application of GreenIT and IT for Green principles in addressing real-world challenges.

\subsection{Sustainable Development Goals (SDGs) and Digital Solutions}

\subsubsection{Alignment with SDGs}

The paper's analysis of digital technologies in GHG mitigation aligns with several Sustainable Development Goals (SDGs), particularly SDG 13 (Climate Action). This connection underscores the  focus on how IT solutions can support broader sustainability objectives. The research emphasizes the potential of digital technologies to contribute to these global goals, reinforcing the concept of IT for Green.

\subsubsection{Contributions to Specific SDGs}

Beyond GHG mitigation, the paper's implications extend to other SDGs, such as SDG 7 (Affordable and Clean Energy) and SDG 11 (Sustainable Cities and Communities). By evaluating digital solutions in transportation and energy sectors, the research contributes to understanding how GreenIT can aid in achieving multiple sustainability targets.

\subsection{GreenIT and Environmental Assessment}

\subsubsection{Life Cycle Assessment (LCA) in Evaluations}

Consistent with the  emphasis on holistic assessments, the paper advocates for the use of Life Cycle Assessment (LCA) in evaluating digital solutions. This approach ensures that all environmental impacts of digital technologies, from production to disposal, are considered, thereby promoting a comprehensive understanding of their environmental footprint.

\subsubsection{Multi-Criteria Assessment Methods}

The paper's methodology resonates with the  discussion on multi-criteria assessment methods in GreenIT. By considering various environmental impacts, including GHG emissions, resource use, and energy consumption, the research aligns with the broader perspective of environmental sustainability in IT.

\subsection{Implications for GreenIT and IT for Green}

\subsubsection{Balancing IT Development and Environmental Impact}

The paper's findings highlight the delicate balance between advancing digital technologies and managing their environmental impacts. This aligns with the  theme of finding sustainable pathways for IT development, ensuring that technological progress does not come at the expense of environmental degradation.

\subsubsection{Guiding Future Research and Development}

The research provides valuable insights for future studies in GreenIT and IT for Green. By identifying current limitations and proposing guidelines for more rigorous assessments, the paper sets a direction for future research. This is in line with the  call for continuous innovation and improvement in the field of IT sustainability.

\subsection{Conclusion}

In summary, the paper's scientific contributions are significant in advancing the understanding and application of GreenIT and IT for Green principles. By providing a critical evaluation of digital technologies in GHG mitigation, proposing methodological guidelines, and aligning with global sustainability goals, the research enriches the discourse on environmental sustainability in the realm of IT.

% Analysis and discussion 
\section{Critical Analysis ($\pm 1p$)}

\subsection{Analysis of Paper's Contributions}

\subsubsection{Contribution to GreenIT and IT for Green}

The publication's exploration into the realistic benefits of digital technologies for GHG emissions mitigation is a vital contribution to both GreenIT and IT for Green concepts. While GreenIT primarily focuses on minimizing the environmental impact of IT itself, IT for Green emphasizes using IT to improve environmental sustainability across various sectors. The paper bridges these two areas by assessing the efficacy of digital solutions in reducing GHG emissions, thereby contributing to broader environmental goals.

\subsubsection{Applicability to Real-World Scenarios}

The research's real-world applicability is commendable. It goes beyond theoretical discussions to evaluate actual digital technologies and their potential in real-world scenarios. This practical approach aligns well with the  emphasis on applying theoretical knowledge to real-world challenges in sustainability.

\subsection{Related Work and Its Contextualization}

\subsubsection{Comprehensive Review of Existing Literature}

The paper does an excellent job of reviewing existing literature and contextualizing its research within the broader field of environmental sustainability in IT. This extensive review not only helps in understanding the current state of research but also in identifying gaps that the paper aims to address.

\subsubsection{Integration of Multi-Disciplinary Perspectives}

The inclusion of multi-disciplinary perspectives, particularly the intersection of technology, environmental science, and policy, is a strong point of the paper. This approach reflects the  focus on the interdisciplinary nature of sustainability issues, where solutions often require input from various fields.

\subsection{Limitations and Areas for Improvement}

\subsubsection{Scope of Technology Assessment}

One limitation of the paper is its scope in assessing the impact of digital technologies. While it does provide valuable insights, the focus is somewhat narrow, primarily centered around GHG emissions. As discussed in the lecture, a more holistic approach, considering other environmental aspects like resource depletion, e-waste, and ecological impacts, would provide a more comprehensive understanding of the sustainability of digital technologies.

\subsubsection{Potential for Broader Impact Analysis}

The paper could benefit from a broader analysis of the indirect impacts of digital solutions, such as societal and economic effects. This aspect aligns with the  emphasis on considering all dimensions of sustainability, not just environmental, to truly assess the impact of technology on sustainable development.

\subsection{Future Work Suggestions}

\subsubsection{Expanding the Methodological Framework}

The future work suggested by the paper, including refining and expanding its methodological framework, is crucial. Applying its guidelines to a wider range of digital solutions would validate and potentially improve their utility in diverse contexts. This suggestion resonates with the  call for continuous improvement and adaptation in sustainability research.

\subsubsection{Long-Term Impact Studies}

Another area for future research is long-term impact studies of digital technologies on sustainability. Understanding the long-term effects, both positive and negative, will be essential in aligning IT developments with sustainable development goals, as highlighted in the lecture.

\subsection{Deeper Examination of Environmental Impacts}

\subsubsection{Comprehensiveness in Environmental Impact Analysis}

While the paper's focus on GHG emissions is pertinent, a more comprehensive analysis encompassing a broader range of environmental impacts would be beneficial. This could include factors like water usage, land use changes, and biodiversity impacts. Such an expanded scope would align with the  emphasis on a holistic approach to environmental sustainability in IT.

\subsubsection{Integration of Life Cycle Assessment (LCA)}

Incorporating Life Cycle Assessment (LCA) methodologies could enhance the depth of the paper's analysis. LCA would provide a more nuanced understanding of the environmental impacts of digital technologies throughout their lifecycle, from production to disposal, resonating with the  discussion on the importance of considering the entire lifecycle in sustainability assessments.

\subsection{Socio-Economic Considerations and Sustainability}

\subsubsection{Balancing Technology and Human Factors}

The paper could further explore the balance between technological advances and socio-economic factors. Understanding how digital technologies interact with human behaviors, economic systems, and societal structures is crucial for effective GHG mitigation strategies, as highlighted in the lecture series.

\subsubsection{Addressing Equity and Accessibility}

Future work should also consider the equity and accessibility aspects of digital solutions for GHG mitigation. Ensuring that these technologies are accessible and beneficial across different socio-economic groups aligns with the comprehensive view of sustainability discussed in the lecture, where social equity is a key pillar.

\subsection{Policy Implications and Recommendations}

\subsubsection{Engagement with Policy and Regulatory Frameworks}

The paper's findings have significant policy implications. Future research could focus on translating these findings into actionable policy recommendations, thus bridging the gap between research and policy-making, a topic emphasized in the lecture series.

\subsubsection{Collaboration with Stakeholders}

Collaborating with a range of stakeholders, including government bodies, industry players, and civil society, can enhance the impact of the research. Such collaboration aligns with the  focus on multi-stakeholder engagement as a critical component of sustainable development.

\subsection{Technological Innovation and Sustainable Development}

\subsubsection{Exploring Emerging Technologies}

The exploration of emerging technologies and their potential role in GHG mitigation is an area ripe for future research. Investigating innovative technologies aligns with the  emphasis on leveraging technological advancements for sustainable development.

\subsubsection{Consideration of Technological Limitations and Risks}

It's also crucial to critically assess the limitations and potential risks associated with deploying new technologies for environmental sustainability, as discussed in the lecture. This includes understanding the trade-offs and unintended consequences that might arise.

\subsection{Conclusion}

The publication provides a valuable contribution to the field of environmental sustainability in IT, aligning well with the concepts discussed in the lecture series. However, expanding its scope to include a more comprehensive environmental impact analysis, considering socio-economic factors, and engaging more deeply with policy implications would enhance its relevance and applicability.

\subsection{Future Research Directions}

\subsubsection{Advancing Methodological Rigor}

Future research based on this paper's findings should aim for greater methodological rigor, especially in scenario modeling and impact assessment. Incorporating robust, diverse, and realistic scenarios, as emphasized in the lecture, can lead to more accurate and meaningful insights into the role of digital technologies in GHG mitigation.

\subsubsection{Interdisciplinary Approaches}

The paper highlights the need for interdisciplinary research combining IT, environmental science, economics, and social sciences. Such an approach resonates with the  emphasis on cross-disciplinary collaboration for comprehensive sustainability solutions.

\subsection{Limitations of the Study}

\subsubsection{Scope of the Research}

While the paper's focus on digital solutions for GHG mitigation is timely, its narrow focus on certain aspects of environmental impacts limits its applicability. Future studies should expand the scope to include a wider range of environmental and social impacts, in line with the  broader perspective on sustainability.

\subsubsection{Data and Methodological Constraints}

The reliance on specific datasets and methodologies may limit the generalizability of the paper's findings. Future work should explore diverse data sources and methodologies, as suggested in the lecture, to enhance the robustness and applicability of the research.

\subsection{Practical Implications}

\subsubsection{Implementing Sustainable IT Practices}

The paper's findings have practical implications for implementing sustainable IT practices in various sectors. This aligns with the  discussion on the practical application of sustainability principles in the IT industry.

\subsubsection{Guidance for Practitioners and Policymakers}

The research provides valuable insights for practitioners and policymakers. Future work should focus on translating these insights into actionable strategies and policies, as underscored in the lecture, to effectively harness digital technologies for environmental sustainability.

\subsection{Conclusion}

Overall, the paper makes a significant contribution to understanding the role of digital technologies in GHG emissions mitigation. However, by addressing its limitations and incorporating the  comprehensive approach to sustainability, future research can provide more nuanced and actionable insights. This will be crucial in leveraging IT for environmental sustainability and achieving broader sustainable development goals.

\section*{References}

\begin{enumerate}
  \item Rasoldier, A., Combaz, J., Girault, A., Marquet, K., \& Quinton, S. (2022). How realistic are claims about the benefits of using digital technologies for GHG emissions mitigation? In LIMITS '22: Workshop on Computing within Limits, June 21–22, 2022.
  \item Brundtland, G. H. (1987). Report of the World Commission on Environment and Development: Our Common Future. United Nations.
  \item United Nations. (1972). Declaration of the United Nations Conference on the Human Environment.
  \item United Nations Environment Programme (UNEP). (2015). Sustainable Development Goals (SDG).
  \item Calero, C., \& Piattini, M. (2015). Introduction to Green in Software Engineering. Green IT.
  \item Murugesan, S. (2008). Harnessing Green IT: Principles and Practices. IT Professional, 10(1), 24-33.
  \item European Commission. (2019). The European Green Deal, COM(2019) 640 final.
  \item Luxembourg National Research and Innovation Strategy. (2020). NATIONAL RESEARCH AND INNOVATION STRATEGY FOR LUXEMBOURG 2020-02.
  \item Climate Watch. (2019). Global GHG Emissions Data.
  \item International Standards Organization (ISO). (2006). ISO 14040:2006 - Environmental management - Life cycle assessment - Principles and framework.
  \item International Standards Organization (ISO). (2006). ISO 14044:2006 - Environmental management - Life cycle assessment - Requirements and guidelines.
\end{enumerate}

% Appendix 
\cleardoublepage
\section*{Appendix ($\pm 2p$)}
\subsection*{Additional Material}

\subsubsection*{Figure 1: Greenhouse Effect Illustration}
\textit{An illustration demonstrating the greenhouse effect, a natural phenomenon essential for life on Earth. Source: Adobe Stock.}

\subsubsection*{Figure 2: Sustainable Development Goals (SDG)}
\textit{Graphic representation of the 17 Sustainable Development Goals set by the United Nations in 2015, focusing on various global challenges including environmental sustainability.}

\subsubsection*{Figure 3: Green IT and IT for Green}
\textit{A diagram explaining the two main aspects of sustainability in relation to IT: 'Green IT' focusing on the sustainability of IT itself, and 'IT for Green' emphasizing IT solutions that enhance environmental sustainability.}

\subsubsection*{Figure 4: Life Cycle Assessment (LCA) Categories}
\textit{A chart depicting the 16 indicators used in Life Cycle Assessment (LCA) for evaluating the environmental impact of products and services. Source: "Understanding and PEF and OEF methods" (European Commission, 2021).}

\subsubsection*{Table 1: Electricity Mix Production in EU and Luxembourg}
\textit{Data table comparing the composition of electricity mix production in the European Union and Luxembourg, detailing the percentages of non-renewable fossil fuels, renewable energy sources, and nuclear energy. Source: ILR, Luxembourg Regulatory Institute.}


%plagiarism section - mandatory - do not remove, do not modify the content
\section*{Plagiarism statement}
\emph{This section is mandatory without modifications.}

I declare that I am aware of the following facts:
\begin{itemize}
    \item I understand that in the following statement the term "person" represents a human or \textbf{\textcolor{red}{ANY AUTOMATIC GENERATION SYTEM}}.
	\item As a student at the University of Luxembourg I must respect the rules of intellectual honesty, in particular not to resort to plagiarism, fraud or any other method that is illegal or contrary to scientific integrity.
	\item My report will be checked for plagiarism and if the plagiarism check is positive, an internal procedure will be started by my tutor. I am advised to request a pre-check by my tutor to avoid any issue.
	\item As declared in the assessment procedure of the University of Luxembourg, plagiarism is committed whenever the source of information used in an assignment, research report, paper or otherwise published/circulated piece of work is not properly acknowledged. In other words, plagiarism is the passing off as one’s own the words, ideas or work of another person, without attribution to the author. The omission of such proper acknowledgement amounts to claiming authorship for the work of another person. Plagiarism is committed regardless of the language of the original work used. Plagiarism can be deliberate or accidental.
Instances of plagiarism include, but are not limited to:
\begin{enumerate}
  \item Not putting quotation marks around a quote from another person’s work
  \item Pretending to paraphrase while in fact quoting
  \item Citing incorrectly or incompletely
  \item Failing to cite the source of a quoted or paraphrased work
  \item Copying/reproducing sections of another person’s work without acknowledging the source
  \item Paraphrasing another person’s work without acknowledging the source
  \item Having another person write/author a work for oneself and submitting/publishing it (with permission, with or without compensation) in one’s own name (‘ghost-writing’)
  \item Using another person’s unpublished work without attribution and permission (‘stealing’)
  \item Presenting a piece of work as one’s own that contains a high proportion of quoted/copied or paraphrased text (images, graphs, etc.), even if adequately referenced
\end{enumerate}
Auto- or self-plagiarism, that is the reproduction of (portions of a) text previously written by the author without citing that text, i.e. passing previously authored text as new, may be regarded as fraud if deemed sufficiently severe.
\end{itemize}

\cleardoublepage
%% end-of-plagiarism section


\end{document}
